\documentclass[a4paper,notitlepage]{report}

\newcommand{\authorOne}   {Jinan \textsc{HARB}}
\newcommand{\authorTwo}   {Manuela \textsc{Furlani da Silva Soares}}
\newcommand{\authorThree} {Meriem \textsc{Smaali}}

% Tweaked from:
% https://fr.overleaf.com/latex/templates/template-for-a-project-report-or-memoire/zcgzvcmrsxsb

%====================== PACKAGES ======================

\usepackage[english]{babel} % Change english -> french if needed
\usepackage[utf8]{inputenc}
\usepackage{float}
\usepackage{amsmath}
\usepackage{graphicx}
\usepackage[colorinlistoftodos]{todonotes}
\usepackage{url}
\usepackage{array}
\usepackage{tabularx}
\usepackage{setspace}
\usepackage{abstract}
\usepackage[T1]{fontenc}
\usepackage[top=2cm, bottom=2cm, left=2cm, right=2cm]{geometry}
\usepackage{subfig}
\usepackage{listings}
\usepackage{hyperref}

\usepackage{xcolor}
\usepackage{csquotes}
\usepackage{fancybox}

\newfloat{code}{htbp}{lop}
\floatname{code}{Code}
\DeclareCaptionSubType{code}

\makeatletter
\lst@Key{matchrangestart}{f}{\lstKV@SetIf{#1}\lst@ifmatchrangestart}
\def\lst@SkipToFirst{%
    \lst@ifmatchrangestart\c@lstnumber=\numexpr-1+\lst@firstline\fi
    \ifnum \lst@lineno<\lst@firstline
        \def\lst@next{\lst@BeginDropInput\lst@Pmode
        \lst@Let{13}\lst@MSkipToFirst
        \lst@Let{10}\lst@MSkipToFirst}%
        \expandafter\lst@next
    \else
        \expandafter\lst@BOLGobble
    \fi}
\makeatother

\usepackage[
    backend=biber,
    style=authoryear-icomp,
    style=numeric,
    sortcites,
    natbib=true,
    sorting=none,
    url=false, 
    doi=true,
    eprint=false
]{biblatex}
\addbibresource{report/biblio.bib}

\definecolor{codegreen}{rgb}{0,0.6,0}
\definecolor{codegray}{rgb}{0.5,0.5,0.5}
\definecolor{codepurple}{rgb}{0.58,0,0.82}
\definecolor{backcolour}{rgb}{1,1,1}
\definecolor{orange}{rgb}{1, 0.65, 0}


\lstdefinestyle{common}{
  basicstyle=\footnotesize\ttfamily
}

\lstdefinestyle{boxed}{
  style=common,
  frame=single,
  keywordstyle=\color{orange},% core keywords <<<<<
  keywordstyle={[2]\color{codepurple}}, % built-ins <<<<
  backgroundcolor=\color{backcolour},   
  commentstyle=\color{red},       
  numberstyle=\tiny\color{codegray},
  stringstyle=\color{codegreen},
  breaklines=true,
  keepspaces=true,
  tabsize=2,
  numbers=left,
  numberstyle=\tiny,
  numbersep=10pt,
  matchrangestart=t
}

\lstdefinestyle{asciiart}{
  style=common,
  columns=fixed,
  fontadjust=true,
  basewidth=0.5em
}

\lstdefinestyle{C}{
  language=C,
  style=boxed,
  title=\texttt{\lstname},
  captionpos=b
}

\lstdefinestyle{tex}{
  language=tex,
  style=boxed,
  title=\texttt{\lstname},
  captionpos=b
}

\makeatletter
\newenvironment{asciiart}{% 
\lstset{style=asciiart}\begin{Sbox}}{% Save the content in a box
\end{Sbox}\centerline{\parbox{\wd\@Sbox}{\TheSbox}}}% And output it centered
\makeatother

%====================== INFORMATION ET REGLES ======================

\setcounter{secnumdepth}{4}
\setcounter{tocdepth}{1}

\hypersetup{
  pdftitle = {SEOC APMAN - C Programming},
  pdfsubject = {Data Structures in C},
  pdfkeywords = {C, data structures},
  colorlinks = true,
  pdfstartview={FitH}
}

\graphicspath{{bench/}{report/schema/}} % Where the Makefile puts the plots (+schemas)

\setlength{\parskip}{5pt plus2pt}


%======================== DEBUT DU DOCUMENT ========================

\begin{document}

\author{\authorOne, \authorTwo, \authorThree}
\title{Phelma {SEOC} 2A {APMAN} Report}

\hypersetup{pageanchor=false}
\begin{titlepage}
  \maketitle

  \begin{abstract}

    Your names must appear in alphabetical order.

    You must create a Git repository on \href{https://gitlab.ensimag.fr}{\texttt{gitlab.ensimag.fr}}. Use any of your accounts.

    The name of your repository must include your logins in alphabetical order:

    \hspace{3cm}\texttt{apman\_login1\_login2\_login3}

    Make sure you made your teachers members of your repository. 

    You may write your report in French (see line 6 in \texttt{report/preamble.tex}) or in English.

    This abstract should list what you did, how, the state of your code and your broad conclusions.
  \end{abstract}

\end{titlepage}
\hypersetup{pageanchor=true}

\tableofcontents

\listoffigures
\listof{code}{List of Codes}

%====================== INCLUSION DES PARTIES ======================

\chapter*{Introduction}\label{chap:introduction}

Data structures form the foundation of efficient algorithm design and software systems. The choice of an appropriate data structure—whether for sequential access, associative lookup, or set membership testing—directly impacts program performance, memory usage, and code maintainability. This project presents a comprehensive implementation and evaluation of fundamental data structures within the \texttt{libellul} C library framework.

\section*{Project Scope}

Our work encompasses the design, implementation, testing, and benchmarking of essential data structures for systems programming in C. The \texttt{libellul} library provides generic, template-based implementations that leverage C preprocessor macros to achieve type-safe polymorphism without runtime overhead. This approach enables the creation of reusable, high-performance components suitable for a wide range of applications.

The project consists of several key components:

\subsection*{Data Structure Implementations}

We have implemented a comprehensive suite of fundamental data structures:

\begin{itemize}
    \item \textbf{Dynamic Arrays}: Resizable arrays with amortized constant-time append operations, supporting both strict and geometric allocation strategies. These provide efficient sequential access with automatic memory management.
    
    \item \textbf{Doubly-Linked Deques}: Intrusive doubly-linked lists enabling efficient insertion and removal at both ends. The implementation uses offset-based techniques to embed link fields within user-defined structures.
    
    \item \textbf{Hash Tables}: Multiple collision resolution strategies including:
    \begin{itemize}
        \item \emph{Linear Probing}: Open addressing with sequential probing
        \item \emph{Robin Hood Hashing}: Variance-reducing open addressing
        \item \emph{Closed Addressing}: Chaining with deque-based buckets
    \end{itemize}
    
    \item \textbf{Hash Tries}: Hierarchical hash structures for persistent and immutable associative containers (implementation in progress).
    
    \item \textbf{Maps and Sets}: High-level abstractions built upon the hash table implementations, providing key-value associations and membership testing with customizable hash functions and equality predicates.
\end{itemize}

\subsection*{Testing and Validation}

Each data structure is validated through a comprehensive test suite built on the \texttt{libellul} unit testing framework. Our testing methodology includes:

\begin{itemize}
    \item \textbf{Property-Based Testing}: Verification of fundamental properties (e.g., arrays maintain insertion order, deques support bidirectional iteration, hash tables preserve all inserted keys).
    
    \item \textbf{Contract Testing}: Systematic validation of preconditions and postconditions using assertion-based checks and abort testing.
    
    \item \textbf{Oracle Testing}: Output validation against expected results, with support for both inline and file-based test oracles.
    
    \item \textbf{Memory Safety}: Integration with AddressSanitizer (ASan) and LeakSanitizer (LSan) to detect memory errors and leaks during test execution.
\end{itemize}

\subsection*{Performance Benchmarking}

We conducted systematic performance evaluation of critical components:

\begin{itemize}
    \item \textbf{Array Benchmarks}: Comparison of heap-allocated versus stack-allocated arrays, measuring cache effects and allocation overhead across problem sizes.
    
    \item \textbf{Hash Table Benchmarks}: Comprehensive evaluation of the three collision resolution strategies (linear probing, Robin Hood, closed addressing) across insert, find, and remove operations. Measurements are normalized per element and plotted as functions of problem size ($\log_2 N$) to reveal algorithmic scaling behavior.
\end{itemize}

\section*{Methodology}

Our implementation follows rigorous software engineering practices throughout the development lifecycle. The \texttt{libellul} library employs C preprocessor templating to provide generic data structures without sacrificing type safety or performance. Each data structure is instantiated through macro-based templates that generate specialized code for user-defined types, enabling zero-cost abstraction similar to C++ templates.

The development workflow integrates automated build systems (Make), comprehensive testing with the custom \texttt{unitest} framework, and performance visualization through gnuplot. All implementations are validated under AddressSanitizer instrumentation to ensure memory safety, and tests include both automated regression suites and interactive debugging support through GDB integration.

Benchmarking follows scientific methodology: we pre-generate random data, warm up the system, measure normalized execution times across logarithmically-spaced problem sizes, and visualize results with publication-quality plots. Each benchmark runs multiple iterations to reduce measurement variance, and timing measurements use high-resolution monotonic clocks for accuracy.


\part{Abstract data types}

\chapter{Arrays}\label{chap:arrays}


\section{Design rationale}

The \texttt{array} module provides dynamic arrays while preserving the familiar syntax and
usage of native C arrays. From the user's perspective, an array remains a simple pointer of
type \texttt{T *}:
\begin{lstlisting}[style=C]
int *a = array(int);              /* initially empty */
double *b = array_new(16, double);/* variable-length, 16 slots */
\end{lstlisting}

\subsection{Internal representation}

To support variable-length behaviour, each array embeds a hidden header located immediately
before the user-visible pointer. The allocated memory block is structured as follows:

\begin{center}
\verb|[ array_header_t ][ element 0 ][ element 1 ] ... |
\end{center}

The header stores:
\begin{itemize}
    \item the current logical length,
    \item the allocated capacity,
    \item the element size in bytes,
    \item a flag distinguishing strict (fixed-size) arrays from resizable ones.
\end{itemize}

Recovering the header is done internally by stepping back one header-sized block from the
user pointer. All public operations, in contrast, manipulate only \texttt{T *} values, keeping the
abstraction transparent for users.

\subsection{Memory management strategy}

All memory allocation is handled by the internal constructor:
\begin{lstlisting}[style=C]
void *array_new__(size_t nmemb, size_t size, int strict);
\end{lstlisting}

This function allocates both the header and the initial number of data slots. For dynamic
arrays, resizing follows an amortized allocation strategy: instead of re-allocating on every
insertion, the module grows the capacity by multiplying the requested size by the geometric
factor \texttt{ARRAY\_ALLOC\_GEOM} (written as \(\approx 1.3\)) and enforcing a minimal
starting capacity. This approach reduces the number of allocations and ensures average
constant-time insertion.

When additional capacity is required, the function \texttt{array\_resize\_\_} reallocates the
entire block, updates the header, and returns the new user pointer. In strict mode,
reallocation never increases the capacity; the array remains permanently bounded.

\subsection{High-level operations}

All high-level operations are implemented on top of the internal resizing mechanism.
The public macro \texttt{array\_resize} forwards the call to \texttt{array\_resize\_\_}, taking the
address of the user pointer so it can be updated after reallocation.

Similarly, stack-like operations are provided through macros:
\begin{itemize}
    \item \texttt{array\_push}: increases the logical length by one and writes the new element;
    \item \texttt{array\_pop}: copies the last element to a user variable and decrements the length.
\end{itemize}

Deletion is explicit: \texttt{array\_delete} wraps the internal \texttt{array\_delete\_\_}
function, freeing the entire header+data block and setting the user pointer to \texttt{NULL} to
avoid dangling references.

\section{Array interface}
\label{arrays:interface}

The public interface of the module is shown in
Listing~\ref{arrays:interface}. The core constructor is the internal function:

\begin{lstlisting}[style=C]
void *array_new__(size_t nmemb, size_t size, int strict);
\end{lstlisting}

It takes three parameters:
\begin{itemize}
    \item the initial number of elements,
    \item the size in bytes of each element,
    \item a flag indicating whether the array is strict (fixed-size) or resizable.
\end{itemize}

\subsection{Constructors}

Three macros form the user-level API:
\begin{itemize}
    \item \texttt{array\_strict(N, T)} creates fixed-size arrays,
    \item \texttt{array\_new(N, T)} creates resizable arrays with an initial capacity,
    \item \texttt{array(T)} creates an initially empty resizable array.
\end{itemize}

\subsection{Destruction}

The destructor \texttt{array\_delete} is a thin wrapper around \texttt{array\_delete\_\_}.  
It receives the address of the array pointer, frees the underlying block starting at the header,
and sets the pointer to \texttt{NULL}.

\subsection{Queries and updates}

The function \texttt{array\_length} returns the logical length, while the macro
\texttt{array\_is\_empty} simply checks for zero length.

Resizing and stack operations are built on the internal \texttt{array\_resize\_\_} function.
The macro \texttt{array\_resize(array\_ptr, nmemb)} exposes this functionality, correctly
updating the caller's pointer. The macros \texttt{array\_push} and \texttt{array\_pop} provide
convenient stack-like operations while maintaining the abstraction that the user manipulates
only plain pointer types.


\chapter{Lists, Stacks, Queues}\label{chap:linear}
Now for the most basic linear abstract data types after arrays.
In this project I reused the existing building blocks (dynamic arrays and intrusive deques)
and added a small layer of generic code templating to obtain list, stack and queue interfaces
specialized for concrete datum types such as \texttt{int}. The idea is to avoid rewriting the
same logic for each type while still getting type-specific functions like
\texttt{int\_stack\_array\_push()} or \texttt{int\_queue\_pop()}.
\section{List interface}\label{lists:interface}
Lists are implemented as a thin wrapper around the intrusive deque presented in
Chap.~\ref{chap:deques}. The list type is essentially an alias for the deque type, and the list
API is just a set of macros that forward to deque operations (push front/back, pop front/back,
first/last, empty, etc.). This way, we get a convenient “list” interface without introducing a new
data structure: all the hard work is done by the deque module.
\section{Stack interface}\label{stacks:interface}
Stacks are generated by a template header \texttt{libellul/type/stack.h}. Before including it,
the user defines:
\begin{itemize}
\item \texttt{T\_STACK\_TAG} (name tag, e.g.\ \texttt{int\_stack\_array} or
\texttt{int\_stack\_deque});
\item \texttt{T\_STACK\_ELEMENT} (element type, e.g.\ \texttt{int});
\item one backend macro: \texttt{T\_IMPL\_STACK\_ARRAY} or
\texttt{T\_IMPL\_STACK\_DEQUE}.
\end{itemize}
From this, the template generates a consistent API: \texttt{TAG\_new()},
\texttt{TAG\_delete()}, \texttt{TAG\_push()}, \texttt{TAG\_pop()}, \texttt{TAG\_top()},
\texttt{TAG\_length()}, \texttt{TAG\_is\_empty()}. I instantiated two versions for \texttt{int}:
\begin{itemize}
\item \textbf{array-based} (\texttt{int\_stack\_array}): the stack is a dynamic array, using
\texttt{array\_push}/\texttt{array\_pop} at the end;
\item \textbf{deque-based} (\texttt{int\_stack\_deque}): the stack is a deque where the back
is the top, storing each value in a small intrusive node.
\end{itemize}
Both versions are tested in \texttt{test/props/stack.c}, checking basic behaviour and LIFO
order.
\section{Queue interface}\label{queues:interface}
Queues are implemented in a similar way, via a template header \texttt{libellul/type/queue.h}.
For an \texttt{int} queue, I define:
\begin{lstlisting}[style=C]
#define T_QUEUE_TAG int_queue
#define T_QUEUE_ELEMENT int
#include <libellul/type/queue.h>
\end{lstlisting}
The template then provides \texttt{int\_queue\_new()}, \texttt{int\_queue\_delete()},
\texttt{int\_queue\_push()}, \texttt{int\_queue\_pop()}, \texttt{int\_queue\_front()},
\texttt{int\_queue\_back()}, etc. Internally, the queue is always backed by a deque: pushes
go to the back, pops remove from the front, which gives natural FIFO behaviour. The file
\texttt{test/props/queue.c} exercises this API and checks that elements come out in the same
order they were inserted.
\section{Benchmark}
To compare the two stack backends, I added a small benchmark in
\texttt{bench/bench\_stack.c}. It instantiates both \texttt{int\_stack\_array} and
\texttt{int\_stack\_deque}, then, for several sizes \(N\), measures the time needed to push
and pop \(N\) elements, averaged over a few rounds, using the standard C \texttt{clock()}
function. The program prints a simple table (size, array time, deque time) that is redirected
by the Makefile into \texttt{bench/bench\_stack.csv}. A gnuplot script
\texttt{bench/bench\_stack.plt} produces the corresponding plot.
\subsection{Observations and conclusions}
The measurements show that the array-based stack is generally faster, especially for large
\(N\), thanks to contiguous memory and cheap operations at the end of the array. The
deque-based stack has a higher overhead (extra nodes and pointer indirections), but it
integrates nicely with the intrusive deque framework used for lists and queues. In practice,
the array backend is a good default for simple stacks of primitive types, while the deque
backend is attractive when structural uniformity and intrusive linking are more important than
raw speed.

\chapter{Maps and Sets}
\label{chap:maps}

Maps are associative containers that store key-value pairs $(k, v)$, where each key is unique.
The main operations are: checking whether a key exists in the map and, if it does, retrieving
its associated value. In this project, maps are implemented in a **generic way** using code
templating: the same header can generate different concrete map types depending on a few
preprocessor definitions.

\section{Template-based map design}

Before including \texttt{libellul/type/map.h}, the user must specify:
\begin{itemize}
    \item a tag \texttt{T\_MAP\_TAG}, which defines the concrete type name;
    \item the key and value types (\texttt{T\_MAP\_KEY} and \texttt{T\_MAP\_VALUE});
    \item the implementation to use (e.g., linear-probing hashtable);
    \item a hash function for the keys (\texttt{T\_MAP\_HASHFUN}).
\end{itemize}

The header expands into a concrete map type and all its associated operations, with full type
safety and without any \texttt{void*} or function-pointer indirection. This approach allows
both performance and safety, since the compiler knows the exact types and can inline helper
functions such as the hash function.

\section{Maps in practice}

Sets are implemented on top of maps by specialising the template to a single element type.
A set of elements $E$ is represented as a map where keys are in $E$ and the values are
ignored. This reuses the underlying hashing infrastructure (buckets, collisions, resizing)
and avoids code duplication.

On top of the basic operations (insert, remove, membership, size), the set abstraction can
provide higher-level operations such as:
\begin{itemize}
    \item union,
    \item intersection,
    \item symmetric difference.
\end{itemize}
All of these are implemented using map operations and iteration over the underlying buckets.

\section{Map interface}

The map interface, defined in \texttt{libellul/include/libellul/type/map.h}, exposes a
minimal and uniform API:
\begin{itemize}
    \item creation and deletion,
    \item insertion (\texttt{put} or \texttt{insert}),
    \item lookup (\texttt{get} or \texttt{contains}),
    \item removal (\texttt{remove}),
    \item query for size (\texttt{length}).
\end{itemize}

The actual behavior is delegated to the chosen backend (linear hashtable, closed hashtable,
Robin Hood hashing, etc.) selected via a macro. For our benchmarks, we instantiate a map
\texttt{dict} associating 64-bit integer keys to 64-bit integer values using a
linear-probing hashtable. A simple hash function on \texttt{uint64\_t} is used.

\section{Set interface}

The set interface is a thin wrapper over the map template. Instead of defining both keys
and values, the user only specifies the element type via \texttt{T\_SET\_ELEMENT} and the
corresponding hash function. Including \texttt{libellul/type/set.h} then generates a
concrete set type with a standard API:
\begin{itemize}
    \item creation (\texttt{new}) and deletion (\texttt{delete}),
    \item insertion (\texttt{insert}) and removal (\texttt{remove}),
    \item membership test (\texttt{contains}),
    \item cardinality (\texttt{length}).
\end{itemize}

\subsection{Example: integer set}

In our project, we instantiate a set of integers \texttt{int\_set} backed by the linear
hashtable implementation:

\begin{lstlisting}[style=C]
static inline unsigned int_hash(int key) {
    return (unsigned)key;
}

#define T_SET_TAG int_set
#define T_SET_ELEMENT int
#define T_IMPL_HASHTABLE_LINEAR
#define T_MAP_HASHFUN int_hash
#include <libellul/type/set.h>
\end{lstlisting}

This generates the type \texttt{int\_set\_t} along with functions such as:
\texttt{int\_set\_new}, \texttt{int\_set\_delete}, \texttt{int\_set\_insert},
\texttt{int\_set\_remove}, \texttt{int\_set\_contains}, and \texttt{int\_set\_length}.

A small test (\texttt{test\_props/test\_int\_set.c}) verifies the correctness of this set:
\begin{itemize}
    \item a new set is empty,
    \item insertions and removals update the cardinality correctly,
    \item membership tests behave as expected,
    \item deletion resets the pointer to \texttt{NULL}.
\end{itemize}

The test is integrated into the \texttt{make checks} target alongside other property-based
tests of the library.





\part{Data structures: Unsorted data}

\chapter{Deques}\label{chap:deques}

A deque (double-ended queue~\cite{wp:deque}) is a linear data structure that supports $O(1)$ time insertion and removal at both ends, making it a versatile building block for lists, stacks, and queues. In this project, we implemented a generic intrusive deque using a circular doubly-linked structure.

\section{Design rationale}

Our design relies on three main principles:

\paragraph{Intrusive nodes.} Each user-defined structure embeds a \texttt{link\_t} node, which allows any object to be part of multiple deques simultaneously and avoids additional memory allocation for deque elements.

\paragraph{Sentinel node.} A dedicated sentinel (\texttt{head}) node is always present. This simplifies insertion and deletion operations by eliminating special cases for empty or single-element deques, and preserves a circular doubly-linked structure for constant-time access to both ends.

\paragraph{Offset-based element retrieval.} The \texttt{deque\_init} macro calculates the offset of the embedded link in the user structure, allowing retrieval of the full element from a \texttt{link\_t*}. This makes the deque fully generic and compatible with arbitrary user-defined types.

\section{Deque interface}\label{deques:interface}

The implemented interface provides the following operations:

\begin{itemize}
    \item Initialization and reset of the deque.
    \item Insertion at the front (\texttt{push\_front}) and back (\texttt{push\_back}).
    \item Deletion at the front (\texttt{pop\_front}) and back (\texttt{pop\_back}).
    \item Access to the first and last elements.
    \item Retrieval of the current deque length.
\end{itemize}

All operations execute in constant time ($O(1)$) due to the circular doubly-linked design and the use of a sentinel node.

\section{Testing and results}

We validated the deque implementation using a comprehensive unit test suite:

\paragraph{Empty deque behavior.} Resetting the deque produces a valid empty structure. Accessing the front or back of an empty deque correctly returns \texttt{NULL}. Deletion operations on an uninitialized or empty deque trigger the expected error handling.

\paragraph{Push/pop consistency.} Using randomly generated user nodes, sequences of push-back and pop-back operations were tested. First and last element pointers consistently reflected the correct elements, and the deque length was accurately maintained.

\paragraph{Memory and structural correctness.} The sentinel node always correctly points to the first and last elements, circularity is preserved after each operation, and no invalid memory accesses were observed.

\paragraph{Results.} All tests passed successfully. The deque correctly handles insertion, deletion, and access operations, both in normal and edge cases. Its intrusive, generic, and circular design makes it robust, efficient, and fully compatible with the \texttt{libellul} framework.



\chapter{Hashtables}\label{chap:hashtables}




\chapter{Hashtables: Closed Addressing}\label{chap:hashtables:closed}

Closed addressing hashtables handle collisions by storing all colliding key/value pairs in a list (or other linear container) associated with each bucket. An important variant is the Move-to-Front strategy, which moves a key to the front of its collision list after each successful lookup to optimize repeated access to the same key.

\section{Design Rationale}

The design uses an array of \texttt{deque\_t} structures to represent buckets. Each bucket maintains a linked list of entries. This approach preserves insertion order and allows efficient insertions, deletions, and lookups.

Key design points:
\begin{itemize}
    \item Collision handling using linked lists (closed addressing)
    \item Reuse of existing list structures (\texttt{deque\_t})
    \item Optional Move-to-Front optimization for frequent lookups
    \item Templated for generic key/value types using \texttt{\#define} macros
\end{itemize}

\section{Implementation}

The main functions implemented are:

\begin{itemize}
    \item \texttt{new}: Allocate and initialize the hash table
    \item \texttt{delete}: Free all allocated memory
    \item \texttt{insert}: Add a key/value pair or update an existing key
    \item \texttt{find}: Retrieve an entry by key
    \item \texttt{remove}: Remove a key/value pair
    \item \texttt{foreach}: Apply a function to all entries
    \item \texttt{length} and \texttt{bucket\_count}: Query the table size and number of buckets
\end{itemize}

\section{Code Example}

\begin{verbatim}
// Creating a new hash table
T ht = CLOSED_METHOD(new)();

// Inserting a key/value pair
T_MAP_ENTRY entry = { .key = someKey, .value = someValue };
CLOSED_METHOD(insert)(ht, &entry);

// Finding a value
T_MAP_ENTRY* found = CLOSED_METHOD(find)(ht, &someKey);
if (found) { printf("Found value: %d\n", found->value); }

// Removing a key
CLOSED_METHOD(remove)(ht, &someKey);

// Deleting the table
CLOSED_METHOD(delete)(ht);
\end{verbatim}



\chapter{Hashtables: Open addressing}\label{chap:hashtables:open}
\section{Linear Probing Hashtable}\label{hashtables:linear}

We implemented an **open-addressing hashtable** using **linear probing**. In this design, each key is hashed to a home slot. Collisions are resolved by checking the next slots sequentially until either the key is found or an empty slot is reached.

Key points of our implementation:

\begin{itemize}
    \item \textbf{Table structure:} The table has a fixed size (\texttt{TABLE\_SIZE}) and stores keys and values in separate arrays. An empty slot is marked with \texttt{EMPTY\_KEY}.
    
    \item \textbf{Insertion (\texttt{put}):} When inserting a key, the home slot is computed using the hash function. If the slot is occupied, subsequent slots are probed linearly until an empty slot is found. If the key is already present, its value is updated.
    
    \item \textbf{Lookup (\texttt{get}/\texttt{contains}):} The home slot is checked first. Linear probing ensures that any colliding key will be found by scanning sequential slots.
    
    \item \textbf{Removal (\texttt{remove}):} The slot containing the key is cleared by setting it to \texttt{EMPTY\_KEY}. In our linear variant, we do not use tombstones, so removed slots become truly empty.
    
    \item \textbf{Memory management:} Keys and values are dynamically allocated arrays. The \texttt{delete} function frees all memory and sets the map pointer to \texttt{NULL}.
    
    \item \textbf{Genericity:} The implementation is type-generic via the \texttt{map.h} interface. Function names are automatically generated for a given key/value type, e.g., \texttt{dict\_new}, \texttt{dict\_put}, \texttt{dict\_get}, etc.
    
    \item \textbf{Collision handling:} Linear probing may cause clustering, which can affect performance at high load factors, but for moderate usage it works efficiently.
\end{itemize}

\subsection{Tests and Results}

Our tests validated the correct behavior of the linear hashtable:

\begin{enumerate}
    \item Creating a new map and verifying it is empty.
    \item Inserting single and multiple keys, checking presence with \texttt{contains}.
    \item Retrieving values with \texttt{get} and verifying correctness.
    \item Updating existing keys and confirming new values.
    \item Removing keys and ensuring they are no longer present.
    \item Re-inserting keys after deletion.
    \item Deleting the map and confirming memory cleanup.
\end{enumerate}

All tests passed successfully, confirming that our linear probing hashtable is reliable and behaves as expected.


\section{Design rationale}
All our open-addressing variants share the same generic \texttt{map.h} interface
(\texttt{new}, \texttt{delete}, \texttt{length}, \texttt{contains}, \texttt{put}, \texttt{remove}).
What changes between implementations is only the collision and deletion policy
(linear probing, tombstones, backward-shift deletion, Robin-Hood), not the API.
This separation let us:
\begin{itemize}
\item reuse the same test code for all variants;
\item plug each variant into the same benchmark driver;
\item compare their behaviour fairly under identical workloads.
\end{itemize}
\section{Example variant: Tombstone reuse}
In the tombstone-based variant, removing a key does not turn the slot back to
``empty'', but marks it with a special tombstone value. Lookups keep probing
across tombstones so probe chains are not broken, and insertions are allowed to
reuse tombstone slots when they encounter them.
In practice:
\begin{itemize}
\item \textbf{lookup} stops only on a truly empty slot;
\item \textbf{remove} writes a tombstone marker instead of an empty key;
\item \textbf{insert} remembers the first tombstone and uses it if the search
ends on an empty position.
\end{itemize}
This keeps the table correct after deletions, at the cost of accumulating
tombstones if no resizing or compaction is performed.
\section{Example variant: Backward-shift deletion (no tombstones)}
Backward-shift deletion removes keys without tombstones. When a key is deleted,
we scan forward inside the same cluster and shift subsequent entries one slot
backward as long as this preserves the validity of their probe sequence. The
cluster shrinks and the first truly empty slot closes the gap.
The effect is:
\begin{itemize}
\item no tombstone markers are needed at all;
\item lookups see only real keys and empty slots, with compact clusters.
\end{itemize}
Deletions become a bit more expensive (they may move several keys), but the
average probe length is kept under control even after many removes.
\section{Example variant: Robin-Hood hashing + Backward-shift deletion}
The Robin-Hood variant reuses the same open-addressing layout and
backward-shift deletion, but changes how insertions resolve collisions. Each
key has a ``home'' slot (its hash) and an implicit probe distance. When
colliding, a key with a larger probe distance is allowed to \emph{steal} the
better position from a key that is closer to its home slot.
Concretely:
\begin{itemize}
\item insertions compare probe distances and swap keys when the newcomer is
more ``unlucky'';
\item deletions still use backward-shift to keep clusters compact;
\item probe lengths are more even across keys, improving worst-case lookup
time at higher load factors.
\end{itemize}
The Robin-Hood implementation passes the same functional tests as the linear
variant and is exercised by a dedicated benchmark (\texttt{bench\_robinhood})
to compare its cost per operation against the simpler strategies.

\chapter{Hashtries}\label{chap:hashtries}

Les \emph{hashtries} sont des structures de type \emph{trie} à $N$ branches ($2 \leq N \leq 256$ en général), dans lesquelles chaque niveau utilise une portion de la clé pour sélectionner le sous-arbre suivant. Dans le cas où les clés ne sont pas des chaînes de caractères, on ne dispose pas de “lettres” pour guider la navigation : les hashtries utilisent alors des portions de bits de la valeur de hachage de la clé pour naviguer dans l’arbre.  

Notre implémentation s’inspire de ce principe tout en restant cohérente avec l’interface déjà utilisée pour les tables linéaires dans le projet.  

\section{Choix d’implémentation}

Voici les principaux choix que nous avons faits pour notre implémentation :

\begin{itemize}
    \item \textbf{Structure récursive de nœuds :}  
    Chaque nœud peut être soit une feuille, contenant une liste chaînée d’entrées (\texttt{key, value}), soit un nœud interne avec des pointeurs vers ses enfants.  
    \emph{Avantage :} Cela permet d’allouer de la mémoire uniquement là où des clés existent, et d’éviter les tableaux fixes trop grands comme dans une table linéaire.

    \item \textbf{Slicing du hash pour naviguer dans l’arbre :}  
    Nous avons défini un nombre fixe de bits par niveau (\texttt{BTREE\_BITS}) et utilisons ces bits successifs du hash de la clé pour sélectionner l’enfant correspondant.  
    \emph{Avantage :} Une distribution uniforme des clés et un accès déterministe à chaque niveau.

    \item \textbf{Gestion des collisions par liste chaînée dans les feuilles :}  
    Les clés qui aboutissent sur la même feuille sont stockées dans une liste chaînée.  
    \emph{Avantage :} Simplifie l’insertion et la suppression sans devoir réorganiser l’arbre entier.

    \item \textbf{Insertion récursive :}  
    L’insertion descend dans l’arbre en utilisant les slices de bits et ajoute la clé dans la feuille correspondante. Si la clé existe déjà, sa valeur est mise à jour.  

    \item \textbf{Suppression récursive :}  
    La suppression suit le même parcours que l’insertion et supprime la clé de la liste de la feuille. Si une feuille devient vide, elle est libérée. Cette approche maintient l’arbre compact.

    \item \textbf{API cohérente avec la table linéaire :}  
    Nous avons conservé les mêmes prototypes de fonctions (\texttt{map\_new}, \texttt{map\_put}, \texttt{map\_get}, \texttt{map\_contains}, \texttt{map\_remove}, \texttt{map\_delete}) pour faciliter l’intégration et les tests.

    \item \textbf{Simplicité et lisibilité :}  
    Nous n’avons pas implémenté de promotion automatique de feuilles vers nœuds internes pour redistribuer les entrées, afin que le code reste simple et facile à comprendre, tout en étant fonctionnel.
\end{itemize}




\section{Illustration et exemple}

\subsection*{Schéma simplifié d’un hashtrie}

\begin{verbatim}

           [Root]
         /    |    \
       000    001   010 ...
       /        \
   [Leaf]       [Leaf]
  (k1,v1)       (k2,v2)
\end{verbatim}

Chaque niveau lit un petit nombre de bits du hash pour choisir le sous-arbre. Les collisions sont gérées dans les feuilles par des listes chaînées.

\subsection*{Exemple d’insertion et recherche}

\begin{enumerate}
    \item Insertion de la clé $k_1$ : le hash de $k_1$ est calculé, les premiers bits sont utilisés pour sélectionner la feuille, et l’entrée $(k_1,v_1)$ est ajoutée à la liste.
    \item Insertion de la clé $k_2$ : si elle tombe sur la même feuille que $k_1$, elle est ajoutée en tête de la liste de collisions.
    \item Recherche de $k_1$ : le hash de $k_1$ est recalculé, les bits successifs déterminent le chemin, et la liste de la feuille est parcourue jusqu’à trouver $k_1$.
    \item Suppression de $k_1$ : même parcours, suppression de l’entrée dans la liste. Si la feuille devient vide, elle est libérée.
\end{enumerate}
\subsection{Problème rencontré lors du benchmark du Hashtrie}

Au cours de l'implémentation et des tests de performance de la structure \textit{Hashtrie}, nous avons rencontré un problème majeur lors de l'exécution du benchmark associé. Le programme \texttt{bench\_hashtrie.bench} se compile correctement et se lance normalement, mais il plante systématiquement lors de l'exécution, généralement aux alentours de \texttt{N = 32768}, sans produire le fichier CSV attendu.

\subsubsection*{Symptômes observés}
\begin{itemize}
    \item Le benchmark démarre correctement en affichant la taille actuelle : \texttt{Hashtrie N = 32768}.
    \item Le programme s'interrompt brutalement (segmentation fault ou crash silencieux).
    \item Aucun message explicite n’est affiché, rendant le débogage difficile.
\end{itemize}

\subsubsection*{Tentatives de résolution}

Afin de résoudre ce problème, plusieurs approches ont été testées :

\begin{enumerate}
    \item \textbf{Analyse du code source existant.}  
    Nous avons inspecté l’implémentation proposée du Hashtrie dans \texttt{hashtrie.h}, en vérifiant notamment les fonctions \texttt{insert}, \texttt{find} et \texttt{remove}, ainsi que la gestion de la récursion et des nœuds. Cette analyse a confirmé que le problème pouvait provenir d’une profondeur de récursion excessive ou d’une gestion incorrecte des nœuds intermédiaires.

    \item \textbf{Modification de \texttt{BTREE\_BITS}.}  
    Plusieurs valeurs ont été testées (\texttt{3}, \texttt{4}, \texttt{5}) afin de réduire la profondeur de l’arbre et limiter les appels récursifs. Bien que cela modifie la structure interne du Hashtrie, le benchmark continue de planter.

    \item \textbf{Réécriture expérimentale de la fonction \texttt{insert}.}  
    Une version itérative de l'insertion (\textit{student-written}) a été développée pour remplacer la version récursive. Cette version limite explicitement la profondeur à \texttt{MAX\_LEVELS} et gère la création des nœuds au fur et à mesure. Bien que fonctionnelle sur de petits tests unitaires, elle ne permet pas d'éliminer entièrement le crash pendant le benchmark intensif.

    \item \textbf{Ajout de traces et tests complémentaires.}  
    Des traces d'exécution (\texttt{printf}) ont été insérées afin de suivre le comportement avant le crash. Des tests supplémentaires ont également été effectués, comme la réinitialisation explicite des pointeurs enfants, et une analyse mémoire préliminaire via \textit{Valgrind}. Malgré cela, l'erreur persistait.
\end{enumerate}

\subsubsection*{Conclusion}
Malgré toutes les tentatives (modifications paramétriques, réécriture de la fonction d'insertion, ajout de limites de sécurité et analyses de comportement), il n’a pas été possible de stabiliser entièrement l’exécution du benchmark pour le Hashtrie. Le problème semble lié à une interaction complexe entre l’implémentation de la structure et le framework de benchmarking utilisé dans le projet.

Cette difficulté a été incluse dans notre rapport comme une limite rencontrée durant le développement, en documentant précisément les essais effectués et les efforts entrepris pour tenter de la résoudre.


\section{Conclusion}

Cette implémentation de hashtrie combine la flexibilité d’un trie avec la simplicité d’une table linéaire. Les collisions sont gérées efficacement via des listes chaînées, et l’API reste cohérente avec le reste du projet. Le code est simple à lire, à maintenir et permet de comprendre facilement la logique des hashtries.



\part{Data structures: Sorted data}

\chapter{Binary Search}\label{chap:binary:search}

Binary search is a classical technique used on data structures that maintain their elements in sorted order. In our case, we implement a \emph{Binary Search Map}, which stores keys (and optionally values) inside a sorted array, enabling logarithmic-time searches. Insertions and deletions remain in linear time due to array shifting, but the simplicity and cache efficiency of arrays make this implementation surprisingly competitive for mid-sized datasets.

This chapter describes our design choices and the implementation details of our student-written binary search map.

\section{Design rationale}

Our objective was to provide a simple and efficient map structure based on sorted arrays while respecting the Libellul generic map interface (\texttt{T\_MAP\_INTERFACE}, \texttt{MAP\_METHOD(...)}). The main design decisions were:

\begin{itemize}
    \item \textbf{Sorted array as core representation.}  
    We store all keys (and values) in ascending order, which allows fast lookups using classic binary search.

    \item \textbf{Dynamic array with amortized growth.}  
    The structure starts with a small initial capacity (\texttt{INITIAL\_CAPACITY = 16}), and doubles its capacity when full. This strategy is standard and ensures amortized $O(1)$ growth.

    \item \textbf{No duplicates allowed.}  
    Since this is a map, duplicate keys are forbidden. If an insertion is performed with a key already present, we simply overwrite the value.

    \item \textbf{Binary search for both lookup and insertion point.}  
    We implemented two internal helpers: one to find an existing key, and another to find the appropriate insertion position in case the key is not found.

    \item \textbf{Cache-friendly layout.}  
    Arrays ensure tight memory packing and therefore better cache locality compared to pointer-based structures such as trees.

\end{itemize}

\section{Binary Search Map Interface}

Our implementation follows the structure expected by Libellul: a generic \texttt{T} structure (the map instance), and a collection of functions such as \texttt{new}, \texttt{put}, \texttt{get}, and \texttt{remove}.

\subsection{Internal structure}

Each map is represented internally as:

\begin{itemize}
    \item a \texttt{length} tracking the number of stored keys,
    \item a \texttt{capacity} controlling memory allocation,
    \item an array of sorted keys,
    \item an optional array of corresponding values (absent in set mode).
\end{itemize}

The structure dynamically resizes when needed, using \texttt{realloc()} to grow while keeping existing keys sorted.

\subsection{Binary search helpers}

We implemented two crucial helper functions:

\begin{description}
    \item[\texttt{bs\_find}]
    Performs a classic binary search and returns the index of the key if found, or \texttt{-1} if absent.

    \item[\texttt{bs\_find\_insert\_pos}]
    Also performs a binary search, but returns the position where the key should be inserted to preserve sorted order.
\end{description}

These two functions ensure that all map operations maintain the ordering invariant.

\subsection{Insertion}

Insertion proceeds as follows:

\begin{enumerate}
    \item Check whether the key already exists using \texttt{bs\_find}.  
          If yes, overwrite its value.
    \item If the array is full, double its capacity using \texttt{realloc()}.
    \item Use \texttt{bs\_find\_insert\_pos} to determine the correct location.
    \item Shift all elements to the right to create a space.
    \item Insert the new key (and value).
\end{enumerate}

Although shifting elements is $O(n)$, this is unavoidable for sorted arrays. For moderate sizes, the cost remains low and cache-friendly.

\subsection{Lookup}

Lookup uses binary search and runs in $O(\log n)$ time.  
If the key is found, the associated value is returned.

\subsection{Removal}

To remove a key:

\begin{enumerate}
    \item Find the key using \texttt{bs\_find}.  
          If absent, return \texttt{0}.
    \item Shift all subsequent elements one position to the left.
    \item Decrease the logical length.
\end{enumerate}

As with insertion, this shifting is necessary to keep the array sorted.

\subsection{Memory management}

The \texttt{delete} method frees:

\begin{itemize}
    \item the keys array,
    \item the values array (when applicable),
    \item the map structure itself.
\end{itemize}

This ensures no memory leaks.

\section{Summary}

This implementation offers:

\begin{itemize}
    \item Fast $O(\log n)$ lookups thanks to binary search,
    \item Predictable and compact memory layout,
    \item Clean integration with the Libellul generic map interface,
    \item Simple, reliable, and efficient behavior for datasets of small to medium size.
\end{itemize}

While insertion and deletion remain linear due to shifting, the trade-off in simplicity and performance for typical workloads makes this approach very practical.


\chapter{Binary search trees}\label{chap:binary:search:trees}

We didn't do it ! We didn't have time to do it 


\chapter{Heaps, priority queues}\label{chap:heaps}

Given an ordering on its $N$ elements (in the form of a comparison callback), a heap is a tree data structure that ensures $O(1)$ access to the minimum and $O(\log N)$ time insertions and removals. This is the basic data structure for implementing a fast $O(N\log N)$ \texttt{heapsort()} sorting routine or efficient priority queues.

Quite likely, you have been using binary heaps in the past: they fit nicely into an array by means of implicit parenting indexing over a complete binary tree, and you have implemented your own \texttt{heapsort()} routine shortly after. Binary heaps lead to fast \texttt{heapsort()} implementations because they use an array as their underlying data structure. 

Nearly all other heap variants use pointer-based (mostly binary) tree representations: rank-pairing heaps, Fibonacci heaps, \textit{etc.} Calling the allocator and constantly jumping to random addresses in memory hide a pretty high constant in the asymptotic running time of these heaps, hence they lead to slow sorting routines. However, they really shine when implementing priority queues because they exhibit much faster, logarithmic-time higher-level routines for \textit{merging} heaps (\textit{that} makes relevant priority queues), an operation that may prove instrumental in quite a few applications---but not in computing a shortest path in a graph\dots Their implementations are full of neat (sometimes nasty!) recursive tricks. 

Overall, the best bet for a good, versatile starting point is to have a generic binary heap on top of a variable-length array. Only merging two priority queues will be a sub-optimal operation. And you would still get a bunch of optimal, important algorithms on graphs in the end.

Remove these paragraphs when you are done!

\section{Design rationale}

\section{Binary heap interface}

\section{Priority queue interface}

\section{Example variant: Rank-pairing heaps}



\chapter{Array sorting}\label{chap:sorting}

We didn't do this part 




\part{Benchmarks}

\section*{Benchmark design}

To evaluate the performance of our data structures, we measure the execution time and memory usage of common operations under controlled conditions. Arrays and maps are tested with a range of sizes using randomly generated values to ensure fair and representative workloads. All benchmarks are repeated multiple times to compute average costs per operation, and memory allocation is tracked via wrapper functions to monitor resource usage reliably. This approach allows comparing different implementations on an equal footing, independently of external factors such as hashing cost or system allocation behavior.


\chapter{Maps for unsorted data}\label{chap:maps:unsorted}

Hash tables are a standard data structure for storing key/value pairs when order is not important. We implemented three variants:

\begin{enumerate}
    \item Closed Addressing
    \item Linear Probing
    \item Robin Hood Hashing
\end{enumerate}

\section{Implementation}

We implemented a generic C hashtable using the \texttt{libellul} library. Keys and values are 64-bit integers, and the table supports insertion, lookup, and deletion.

\subsection{Closed Addressing}

\begin{verbatim}
// Example C code snippet for closed addressing
T ht = dict_new();
entry_t e = {.key = 42, .value = 100};
link_init(&e.link);
dict_insert(ht, &e);
T_MAP_ENTRY *found = dict_find(ht, &e.key);
dict_remove(ht, &e.key);
dict_delete(ht);
\end{verbatim}

\subsection{Linear Probing}

\begin{verbatim}
// Similar setup for linear probing map
dict_t map = dict_new();
dict_put(&map, 42, 100);
uint64_t val;
dict_get(map, 42, &val);
dict_remove(&map, 42);
dict_delete(&map);
\end{verbatim}

\subsection{Robin Hood Hashing}

\begin{verbatim}
// Robin Hood variant example
dict_t map = dict_new();
dict_put(&map, 42, 100);
uint64_t val;
dict_get(map, 42, &val);
dict_remove(&map, 42);
dict_delete(&map);
\end{verbatim}

\section{Benchmarking}

We benchmarked insertion, lookup, and deletion for varying sizes of keys. Each operation was timed over multiple runs to obtain average cost per operation.

\begin{verbatim}
// Benchmark code snippet
benchmark_insert(N, RUNS);
benchmark_find(N, RUNS);
benchmark_remove(N, RUNS);
\end{verbatim}


\chapter{Maps for sorted data}\label{chap:maps:sorted}




\chapter{Array Performance Benchmarks}\label{chap:benchmark:sorting}

Before benchmarking sorting routines themselves, it is instructive to evaluate the performance characteristics of the underlying dynamic array implementation, as sorting relies heavily on element access and manipulation. We measure the time taken for common array operations, including allocation, insertion, reading, writing, and removal.

\section{Benchmark Setup}

All benchmarks were performed using the dynamic array implementation described in Chap.~\ref{chap:heaps}. The following functions were benchmarked:

\begin{itemize}
    \item \texttt{array\_new(N, int)} -- allocate an array of size \texttt{N}.
    \item \texttt{array\_push(\&arr, value)} -- append an element to the array.
    \item Random reads from the array.
    \item Random writes to the array.
    \item \texttt{array\_pop(\&arr, \&x)} -- remove the last element.
\end{itemize}

The benchmarks were performed on arrays of size $N = 2^{\text{log2\_N}}$ with \texttt{log2\_N} ranging from \texttt{LOG2\_N\_MIN} to \texttt{LOG2\_N\_MAX}. Each test was repeated \texttt{RUNS} times to compute an average execution time in nanoseconds.

\section{Benchmarking Methodology}

A high-resolution timer (\texttt{elapsed\_nsec()}) was used to measure the duration of each operation. For reads, a \texttt{volatile int sink} variable was used to prevent compiler optimizations. For writes, random indices were used to simulate non-sequential access patterns.  

The benchmarks include:

\begin{enumerate}
    \item \textbf{Allocation (new)}: time to allocate an array of size $N$.
    \item \textbf{Push}: time to sequentially append $N$ elements to an initially empty array.
    \item \textbf{Read}: time to access $N$ random elements.
    \item \textbf{Write}: time to write to $N$ random positions.
    \item \textbf{Pop}: time to remove all elements sequentially.
\end{enumerate}

\section{Results}

Table~\ref{tab:array-benchmarks} summarizes the measured times.  

\begin{table}[h!]
\centering
\caption{Average execution times (ns) for array operations.}
\begin{tabular}{r|r|r|r|r|r}
\textbf{log2(N)} & \textbf{N} & \textbf{new} & \textbf{push} & \textbf{read} & \textbf{pop} \\
\hline
0  & 1      & 37.9     & 36.5       & 38.7       & 23.4     \\
1  & 2      & 27.2     & 57         & 49.3       & 25.1     \\
2  & 4      & 25.4     & 73.6       & 78.3       & 32.4     \\
3  & 8      & 25.9     & 157        & 136        & 47.7     \\
4  & 16     & 25.5     & 207        & 253        & 82.2     \\
5  & 32     & 26.5     & 377        & 487        & 157      \\
6  & 64     & 29.2     & 582        & 956        & 303      \\
7  & 128    & 26.2     & 869        & 1.9e+03    & 567      \\
8  & 256    & 36.7     & 1.45e+03   & 4.46e+03   & 1.11e+03 \\
9  & 512    & 155      & 2.83e+03   & 7.49e+03   & 3.5e+03  \\
10 & 1024   & 41.9     & 7.96e+03   & 1.84e+04   & 4.55e+03 \\
11 & 2048   & 292      & 1.07e+04   & 3.65e+04   & 1.29e+04 \\
12 & 4096   & 288      & 3.39e+04   & 6.82e+04   & 2.5e+04  \\
13 & 8192   & 892      & 5.01e+04   & 1.39e+05   & 3.68e+04 \\
14 & 16384  & 314      & 7.9e+04    & 2.58e+05   & 7.7e+04  \\
15 & 32768  & 1.31e+03 & 1.72e+05   & 5.04e+05   & 1.39e+05 \\
\end{tabular}
\label{tab:array-benchmarks}
\end{table}


Figure~\ref{fig:array-performance} visualizes the scaling behavior of these operations.

\begin{figure}[h!]
\centering
\includegraphics[width=0.8\textwidth]{bench/array_performance.png}
\caption{Benchmark results for dynamic array operations. Allocation, push, and pop times are shown as a function of array size.}
\label{fig:array-performance}
\end{figure}

\section{Discussion}

The results demonstrate the expected time complexity of common array operations:

\begin{itemize}
    \item Allocation (\texttt{new}) scales linearly with the number of elements.
    \item Sequential \texttt{push} operations are efficient, benefiting from amortized constant-time insertions due to dynamic resizing.
    \item Random reads and writes show constant-time access as expected for a contiguous memory array.
    \item Sequential \texttt{pop} operations mirror the push behavior in reverse, confirming efficient memory management.
\end{itemize}

These benchmarks provide a baseline for subsequent sorting experiments. Since sorting routines frequently read, write, and swap elements, understanding the raw array performance helps explain the observed sorting times.



\appendix

\chapter{Technical notes}

You may want to save supporting technical information here, so that you can refer to it from multiple places.

\section{CPU design}

\section{Memory}

\section{C language}

\section{Compilers}





%======================== WHO DID WHAT ========================

\chapter{Who did what?}

%-------------------------------------------------------------
\section{Jinan HARB}

\subsection{What I did}

\begin{itemize}
  \item Implemented arrays and their tests (\texttt{01-array.c}) — collaborative work.
  \item Implemented deques and their tests (\texttt{02-deques.c}) — collaborative work.
  \item Developed closed-addressing hashtables (\texttt{hashtables\_closed.c}) and their associated tests (\texttt{test\_closed.c}).
  \item Implemented hashtable benchmark programs: 
    \begin{itemize}
      \item \texttt{bench\_closed.c}
      \item \texttt{bench\_robinhood.c}
      \item \texttt{bench\_linear.c}
      \item \texttt{benchmark.h}
      \item Array benchmark: \texttt{bench\_array.c}
    \end{itemize}
  \item Added my contributions to the written report.
\end{itemize}

\subsection{What I learned}

This project really helped me get a deeper understanding of coding in C. Implementing arrays, deques, and different types of hashtables made me see how different data structures actually work and how their design affects performance. Implementing benchmarks taught me how to measure efficiency, find performance issues, and test code properly.

I also learned a lot about organizing code, using headers and separate files to keep things clean.  

On top of coding, being part of the project coordination and contributing to the report helped me improve how I plan work, communicate with the team, and keep everyone on the same page. Overall, I feel like I really leveled up both in practical coding skills and in understanding how to structure and manage a programming project.


%-------------------------------------------------------------
\section{Manuela Furlani da Silva Soares}

\subsection{What I did}

\begin{itemize}
  \item Implemented arrays and their tests (\texttt{01-array.c}) — collaborative work.
  \item Implemented deques and their tests (\texttt{02-deques.c}) — collaborative work.
  \item Implemented the Robin-Hood open-addressing hashtable variant and its tests.
  \item Implemented the concrete \texttt{int\_set} type on top of the generic map/set templates and wrote a small test suite for it.
  \item Implemented linear containers on top of existing primitives:
    \begin{itemize}
      \item stacks from dynamic arrays;
      \item stacks and queues from intrusive deques;
      \item tests for these structures in \texttt{test\_linear.c}.
    \end{itemize}
  \item Implemented the benchmarks to the linear part.
  \item Implemented a benchmark comparing stacks implemented with dynamic arrays and stacks implemented with deques (same timing infrastructure as the hashtable benchmarks).
  \item Contributed to the written report.
\end{itemize}

\subsection{What I learned}

\begin{itemize}
  \item Learned a lot about language C, since it was my first contact with the language. 
  \item From the collaborative work on arrays, I learned how to design a generic dynamic array in C using a hidden header (storing length, allocated size, element size and a strict flag).
  \item From the collaborative implementation of deques, I learned how intrusive doubly-linked lists work, how to use an anchor node plus an \texttt{offsetof}-based offset to recover user nodes, and how this design gives constant-time operations at both ends.
  \item How to use the generic map/set templates to instantiate concrete types such as \texttt{int\_set}, and to test them systematically.
  \item The main trade-offs between different open-addressing strategies, in particular the behaviour of Robin-Hood hashing compared to simple linear probing.
  \item How to build stacks and queues on top of dynamic arrays and intrusive deques, and how to validate them using the common unit test framework.
  \item How to interpret benchmark results when comparing multiple implementations under the same interface.
\end{itemize}


%-------------------------------------------------------------
\section{Meriem Smaali}

\subsection{What I did}

\begin{itemize}
  \item Collaborative implementation of arrays and their tests (\texttt{01-array.c}).
  \item Collaborative implementation of deques and their tests (\texttt{02-deques.c}).
  \item Implementation of linear probing hashtables and associated tests.
  \item Implementation of the hashtrie structure and its integration into the MAP interface and initial benchmark attempts.
  \item Implementation of the binary search map (\texttt{Binary\_search\_map.h}).
  \item Implemented or contributed to multiple benchmark programs.
  \item Contributed to the writing of the report.
  \item Took the role of project coordinator for the group: organized meetings, managed the workflow, and ensured that the team remained synchronized throughout the project.
\end{itemize}

\subsection{What I learned}

During this project, I learned how to implement several abstract data types and how to structure them cleanly in C. Working with macros and the generic MAP interface helped me understand how to write reusable code without repeating the same implementation for each key/value type. This was new to me and taught me how powerful—and sometimes tricky—the C preprocessor can be.

By implementing arrays, deques, linear probing, open-addressing hashtables, hashtries, and the binary search map, I improved my understanding of memory management, pointer safety, and the importance of choosing the right data structure for each operation. I also learned how to build and run benchmarks, even if I could not fully fix the issue in the hashtrie benchmark despite several debugging attempts.


\printbibliography

\end{document}
